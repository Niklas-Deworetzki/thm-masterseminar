
\section{Futamura Projections}\label{sec:futamura}

As seen before in \ref{TODO} a computation can be performed in different ways and multiple computational stages, for example by using an interpreter or compiler.
Also visible, altough not specifically emphasized, was the fact that the interpreter used to execute the source program does accept two kinds of input, namely the source program and the problem input for the source program.
One might wonder now what would happen if an partial evaluator is used to separate those two inputs.

In ?? Futamura ... presented three projections.


\subsection{The first Futamura projection}\label{sec:futamura1}

The first of Futamura's projections deals with the case mentioned above, where an partial evaluator is used to specialize an interpreter to a source program. % TODO: Does this sentence make sence?
In this case the partial evaluator is used, to specialize an interpreter to a fixed source program.
The resulting residual program now accepts an input to produce the desired output, as if the original source program would be executed.
% TODO: PE can be used to compile a source program.

\begin{align*}
  \mathtt{output}\ &=\ \llbracket \mathtt{source} \rrbracket_{\mathtt{S}}\ \mathtt{input} \\
                   &=\ \llbracket \mathtt{interpreter} \rrbracket_{\mathtt{L}}\ [\mathtt{source},\, \mathtt{input}] \\
                   &=\ \llbracket \llbracket \mathtt{mix} \rrbracket\ [\mathtt{interpreter},\, \mathtt{source}] \rrbracket\ \mathtt{input} \\
                   &=\ \llbracket \mathtt{target} \rrbracket\ \mathtt{input}
\end{align*}


\subsection{The second Futamura projection}\label{sec:futamura2}

Looking at the equations from \ref{sec:futamura2} it is noticeable, that again some program is applied to two different inputs.
This time it is the partial evaluator itself, that is applied to an interpreter and the source program.
The second Futamura projection shows the effects of using a partial evaluator to specialize a partial evaluator to an interpreter.
The resulting residual program from specialization of the partial evaluator to an interpreter is a program, that accepts a source program as an input and creates a target program as an output.
Using a partial evaluator and an interpreter, it is possible to create a residual program acting like a compiler.

\begin{align*}
  \mathtt{target}\ &= \llbracket \mathtt{mix}\rrbracket\ [\mathtt{interpreter},\, \mathtt{source} ] \\
                   &= \llbracket \llbracket \mathtt{mix} \rrbracket\ [\mathtt{mix},\, \mathtt{interpreter}] \rrbracket\ \mathtt{source} \\
                   &= \llbracket \mathtt{compiler} \rrbracket\ \mathtt{source}
\end{align*}


\subsection{The third Futamura projection}\label{sec:futamura3}

The second projection shows, how a partial evaluator and an interpreter can be combined to create an compiler.
But again it is noticeable in the equations, that some program is applied to two different inputs.
In this case, the partial evaluator is applied to itself and an interpreter to create the compiler.
The third Futamura projection shows what happens, if a partial evaluator is used to specialize this application.

Since previously the partial evaluator accepted itself and an interpreter as input, we now use the partial evaluator to specialize itself to itself.
The resulting program accepts an interpreter to create a compiler.
Using nothing more than a partial evaluator and an interpreter, we created a compiler generator.

\begin{align*}
  \mathtt{compiler}\ &=\ \llbracket \mathtt{mix} \rrbracket\ [\mathtt{mix},\, \mathtt{interpreter}] \\
                     &=\ \llbracket \llbracket \mathtt{mix} \rrbracket\ [\mathtt{mix},\, \mathtt{mix}] \rrbracket\ \mathtt{interpreter} \\
                     &=\ \llbracket \text{\texttt{compiler-gen}} \rrbracket\ \mathtt{interpreter}
\end{align*}

Even though the equations still show, that some program is applied to two inputs, it becomes clear why further application of the previous scheme will not yield any new functionality.
Using an partial evaluator to specialize the partial evaluator to one of its inputs, will yield the same equation as before, only introducing an additional stage of computation.


\subsection{Advantages of Self-Application}\label{sec:self-application}

While not yielding any new functionality it may still be desirable to further specialize the partial evaluator.
As shown in \citationneeded{} it is possible to gain significant performance gains, when a specialized program is used for specialization instead of the general partial evaluator.

\begin{align*}
  \mathtt{target}\ =\
  \llbracket \mathtt{mix} \rrbracket\ [\mathtt{interpreter},\, \mathtt{source}]\ &=\
  \llbracket \mathtt{compiler} \rrbracket\ \mathtt{source} \\
  \mathtt{compiler}\ =\
  \llbracket \mathtt{mix} \rrbracket\ [\mathtt{mix},\, \mathtt{interpreter}]\ &=\
  \llbracket \text{\texttt{compiler-gen}} \rrbracket\ \mathtt{interpreter} \\
  \text{\texttt{compiler-gen}}\ =\
  \llbracket \mathtt{mix} \rrbracket\ [\mathtt{mix},\, \mathtt{mix}]\ &=\
  \llbracket \text{\texttt{compiler-gen}} \rrbracket\ \mathtt{mix}
\end{align*}

%%% Local Variables:
%%% mode: latex
%%% TeX-master: "../paper"
%%% End:
