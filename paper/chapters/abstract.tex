
\thispagestyle{empty}
\begin{quote}
  \vspace*{4cm}

  \begin{center}
    \textbf{\Large\sffamily Abstract}
  \end{center}

  \vspace{1em}

  Partial evaluation provides a unifying paradigm for program generation and program analysis.
  The key feature of partial evaluation is the specialization of programs; an optimization technique and execution strategy, that uses fixed inputs to a program to pre-compute parts of it and thereby increasing the program's performance.

  This paper introduces the basic concepts of partial evaluation and classifies them in contrast to classical approaches of program execution and optimization.
  Core concepts of partial evaluation such as the difference between online and offline partial evaluation as well as basic specialization strategies are explained.
  As a key insight, the Futamura projections are presented and explained, providing a direct link between partial evaluation and compiler construction.
  Finally, a critical assessment of partial evaluation is provided, highlighting different use cases from compiler construction to software engineering, while also showing its boundaries.
\end{quote}



%%% Local Variables:
%%% mode: latex
%%% TeX-master: "../paper"
%%% End:
