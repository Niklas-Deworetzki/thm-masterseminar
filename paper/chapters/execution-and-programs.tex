
\section{Execution of Programs}\label{chap:programs}

Usually, we think of the execution of programs as one single step.
Either a program is executed, producing an output to its given inputs, or a program sits on a hard drive without being executed.
This way, we don't have to worry about the exact details of how a program is executed and can focus on the computational problem a program tries to solve.
However, in the following chapters, we will look into many ways to execute a program, all of which lead to the same output but are drastically different in their use cases, advantages and behaviors.

\begin{figure}[h]
  \centering
  \begin{tikzpicture}[
    base/.style={draw, thick, minimum width = 2.4cm, minimum height = 1cm},
    data/.style={rectangle, base},
    program/.style={ellipse, base},
    arrow/.style={-triangle 90}]

    \node[program] (prog) at (0, 0) {program};
    \node[data, right = of prog] (output) {output};
    \node[data, left = of prog] (input2) {input\textsubscript{2}};
    \node[data, above = 0cm of input2] (input1) {input\textsubscript{1}};
    \node[data, below = 0cm of input2] (input3) {$\ldots$};

    \draw [arrow] (input1.east) -- (prog);
    \draw [arrow] (input2.east) -- (prog);
    \draw [arrow] (input3.east) -- (prog);
    \draw [arrow] (prog) -- (output);
  \end{tikzpicture}
  \caption{Direct execution of a program}\label{fig:native-program}
\end{figure}


Figure~\ref{fig:native-program} schematically presents the way, we usually think of the execution of a program.
A program (described by an oval shape) accepts some data (described by boxes) as an input and transforms this data into an output.
While this diagram conceals details of how execution actually takes place, it gives an appropriate intuition about the general flow of data during execution.
The conventional execution of a program takes place as a single computational step.
Even if a program gets more complex, accepting many different inputs and producing output for each of these cases, it is only the run-time that grows, not the computational stages.


The figure, even if useful as an introduction, does not correctly represent reality in most cases.
Usually, a program written by a programmer is not executable directly on a computer and the help of other programs is required to transform the raw source code into an executable format.
These programs are so-called meta-programs, as the datasets they work on are programs as well.
In the context of this work, we will come into contact with three different meta-programs that are used for the execution of other programs, namely interpreters, compilers and partial evaluators.
While interpreters and compilers are usually familiar, the concept of a partial evaluator often seems foreign.
To put this concept in a better perspective, we will start with a brief description of interpreters and compilers before partial evaluators are finally introduced.


\subsection{Interpreters}

Interpreters, as already mentioned are meta-programs that are used to execute other programs.
An interpreter accepts the program, which is to be executed, as an input and analyzes it, whereby it is usually transformed into an intermediate representation.
This tree-like structure mirrors the structure and contents of the source program.
By assigning meaning to every node of this tree, a program becomes executable, as the tree can be traversed with the interpreter performing actions corresponding to the meaning of each node.

Now, given that the interpreter is executable, all programs written in a language the interpreter can \enquote{understand} are executable too.
As Figure~\ref{fig:interpreted-program} shows, the interpreted execution of a program looks very similar to the direct, native execution.
The only difference is, that the program to be executed is now itself passed as an input to the interpreter along with its inputs.
Execution this way is still a single computational stage.

Interpreters provide a simple solution to execute programs in another language and are relatively easy to implement.
On the downside, execution via an interpreter provides less performance, as interpreter and interpreted program share the same resources and some computational overhead is required for analysis of said program.

\begin{figure}[h]
  \centering
  \begin{tikzpicture}[
    base/.style={draw, thick, minimum width = 2.4cm, minimum height = 1cm},
    data/.style={rectangle, base},
    program/.style={ellipse, base},
    arrow/.style={-triangle 90}]

    \node[program] (interpreter) at (0, 0) {interpreter};

    \node[data, above left = -0.4cm and 1.4cm of interpreter] (input) {input\textsubscript{1}};
    \node[data, below = 0cm of input] (input2) {input\textsubscript{2}};
    \node[data, above = 0cm of input] (program) {program}; \node[program, above = 0cm of input] {};
    \node[data, below = 0cm of input2] (input3) {$\ldots$};

    \node[data, right = of interpreter] (output) {output};

    \draw [arrow] (program.east) -- (interpreter);
    \draw [arrow] (input.east) -- (interpreter);
    \draw [arrow] (input2.east) -- (interpreter);
    \draw [arrow] (input3.east) -- (interpreter);
    \draw [arrow] (interpreter) -- (output);
  \end{tikzpicture}
  \caption{An interpreter executing a program}\label{fig:interpreted-program}
\end{figure}


\subsection{Compilers}

Compilers, similar to interpreters, are programs that are used to transform other programs.
While an interpreter uses the intermediate representation of a program to directly executed the assigned meaning to it, a compiler translates the meaning into another language.
As a result, a compiler produces a single output, which is the so-called target program.
This program holds the meaning of the compiler's input~--~the source program~--~translated into another language.
As seen in Figure~\ref{fig:compiled-program}, an additional stage of computation is introduced this way.
The target program has to be executed for it to accept inputs and produce an output like the original source program.

As already apparent visually, the translation process of a compiler is more complex than the process of an interpreter.
But since the input program has to be only analyzed once and the target program is executed as a standalone program, compilation usually provides better performance in contrast to interpretation.
This is especially true if the same program is executed multiple times since no computational overhead is present during the execution of the target program after it has been created once.

\begin{figure}
  \centering
  \begin{tikzpicture}[
    base/.style={draw, thick, minimum width = 2.4cm, minimum height = 1cm},
    data/.style={rectangle, base},
    program/.style={ellipse, base},
    arrow/.style={-triangle 90}]

    \node[program] (compiler) at (0, 0) {compiler};
    \node[data, left = of compiler] (source) {source};
    \node[data, below = of compiler] (target) {target}; \node[program, below = of compiler] {};

    \node[data, left = of target] (input) {input};
    \node[data, right = of target] (output) {output};

    \draw [arrow] (source) -- (compiler);
    \draw [arrow] (compiler) -- (target);

    \draw [arrow] (input) -- (target);
    \draw [arrow] (target) -- (output);
  \end{tikzpicture}
  \caption{A compiler generating an executable target program}\label{fig:compiled-program}
\end{figure}



\subsection{Partial Evaluators}

Partial evaluators, as the third meta-program introduced and focus of this paper, generalize the previously mentioned concept of creating additional computational stages.
The inputs of a partial evaluator are a source program as well as \textit{some} inputs for this program.
Given these inputs, a partial evaluator will perform all computations in the source program that are available under the given inputs.
As not all inputs for the source program are present, some computations cannot be performed.
These remaining computations form a so-called residual program, the output of a partial evaluator.

\begin{figure}[h]
  \centering
  \begin{tikzpicture}[
    base/.style={draw, thick, minimum width = 2.4cm, minimum height = 1cm},
    data/.style={rectangle, base},
    program/.style={ellipse, base},
    arrow/.style={-triangle 90}]

    \node[program] (pe) at (0, 0) {pe};
    \node[data, left = of pe] (input1) {input\textsubscript{1}};
    \node[data, above = 0cm of input1] (source) {source};
    \node[data, below = of pe] (residual) {residual}; \node[program, below = of pe] {};

    \node[data, left = of residual] (input2) {input2};
    \node[data, below = 0cm of input2] (input3) {$\ldots$};
    \node[data, right = of residual] (output) {output};

    \draw [arrow] (source.east) -- (pe);
    \draw [arrow] (pe) -- (residual);

    \draw [arrow] (input1.east) -- (pe);
    \draw [arrow] (input2.east) -- (residual);
    \draw [arrow] (input3.east) -- (residual);

    \draw [arrow] (residual) -- (output);
  \end{tikzpicture}
  \caption{A partial evaluator generating a residual program}\label{fig:partial-evaluated-program}
\end{figure}

While Figure~\ref{fig:partial-evaluated-program}, describing a partial evaluator, is structurally similar to Figure~\ref{fig:compiled-program} describing a compiler, the concept is generalized.
A compiler only splits the overhead of analysis and the actual computation into different computational stages.
A partial evaluator can split calculations depending on different inputs into different computational stages.
This way, a general program accepting multiple inputs can be specialized to a fixed input, which usually leads to higher performance, since the overhead of recognizing inputs is removed.



%%% Local Variables:
%%% mode: latex
%%% TeX-master: "../paper"
%%% End:
