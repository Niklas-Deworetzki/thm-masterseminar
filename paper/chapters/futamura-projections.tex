
\section{The Futamura-Projections}\label{sec:futamura}

Until now, PE was only used as a way to specialize programs.
As seen, specialized program does not have new properties.
Only reduction of general programs, simplifying their structure, optimizing runtime by removing expressions.

Futamura proposed three projections, showing that PE can seemingly create new functionality.
Programs with different properties appear through specialization.
Clever use of interpreters and the way PEs work.

Firstly a new notation is used to simplify his findings.
Notation shows evaluation as equations.

\subsection{Notation of Multistage Computations}\label{sec:multistage-notation}

As seen before, computations can be split into multiple stages.

Program of language S can be executed directly, accepting input and generating output if S is executable.
Interpreter can also be used to perform computations.

Another example: Compiler splitting two stages of computation.
Two nested brackets in equation.

Partial Evaluator works similarly, introducing two nested brackets.


\subsection{The First Futamura Projection}\label{sec:futamura-first}

The first projection is based on a fact, visible before but not directly emphasized.
During multistage computation and intro on execution we see, interpreters accept two inputs.
First input: The executed program.
Second input: The input to the executed program.

Usually two are passed in at once, PE can be used to create multiple stages of computation.
Interpreter is specialized to program, input remains variable.

Resulting residual program accepts the input of the program and produces output of program corresponding to input.
Residual program is compiled program.
PE and interpreter acted like compiler.


\subsection{The Second Futamura Projection}\label{sec:futamura-second}

Looking at previous projection, it can be seen that a program is applied to two inputs at once.
PE accepts interpreter and source program.
The second projection deals with this case and shows what happens if input is separated.

PE is used to specialize PE on interpreter.
Source program remains variable.

Residual program accepts a source program to create target program.
Residual program acts as compiler.
PE and interpreter acted like compiler compiler.


\subsection{The Third Futamura Projection}\label{sec:futamura-thrid}

Again, looking at previous projection, program is applied to two inputs at once.
PE accepts PE and interpreter.
The third projections deals with this case and shows what happens if input is separated.

PE is used to specialize PE on PE.
Interpreter remains variable.

Residual program accepts an interpreter, to create a program.
This program then accepts source to generate target.
Residual program acts as compiler compiler.
PE and PE acted like compiler compiler compiler, requiring only PEs.
Interpreter determines how compiler acts.


\subsection{Is there a Fourth Futamura Projection?}\label{sec:futamura-fourth}

Equation now only consists of PEs.
While additional is possible, it does not seem to change equation itself.


\subsection{Going Further}\label{sec:self-application}

No new functionality seems to be yield.
Its still desirable to further specialize.

Multiple ways to create program are equal on a theoretical level, shown in equations.
Practical differences in generated real code.
As shown, performance gains.


%%% Local Variables:
%%% mode: latex
%%% TeX-master: "../paper"
%%% End:
