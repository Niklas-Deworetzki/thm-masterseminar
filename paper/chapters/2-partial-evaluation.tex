
\section{Instruments of Partial Evaluation}\label{sec:instruments}

In this section we will look into the inner workings of a partial evaluator.
More concrete, this section shows the different methods that can be used by a partial evaluator to specialize programs and what effect they have on real-world examples.

The main goal of an partial evaluator is to evaluate all computations that can be evaluated statically given parts of the programs input.
But how can the partial evaluator decide, which computations can be evaluated statically and for which a residual program has to be generated?
Naturally some kind of analysis is performed on the source program, deciding what computations can be performed.
In~\cite[S.3]{Jones_PartialEvaluation}~three main techniques are presented, that can be used to specialize a program:

\begin{enumerate}
\item Symbolic computation
\item Unfolding of function calls and
\item program point specialization.
\end{enumerate}

In the following, these three techniques are explained individually and examples are given for each one.
% TODO: Optimization where?


\subsection{Unfolding of function calls}

% TODO: Can be used to propagate known values.


\subsection{Symbolic Computation}

The term \enquote{symbolic computation} is a broad term but can be described by arithmetic or algebraic simplifications in regards to partial evaluation.


\citationneeded[Jones 132]


\subsection{Program point specialization}




%%% Local Variables:
%%% mode: latex
%%% TeX-master: "../paper"
%%% End:
