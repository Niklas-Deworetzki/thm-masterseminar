
\section{Multistage Computations}\label{sec:multistage}

When solving computational problems, the required calculations can be performed either in a single computational stage, or can be separated into multiple stages.
This is especially obvious when working with compiled languages, since they require the help of a compiler to generate executable program code, that is able to perform calculations.
Since multistage computations will be a recurring theme in the following sections, we now introduce a notation to reason about it in a formal way. This notation was adopted from . % TODO

Given a program to solve an arbitrary computational program, this program is defined by its source code written in a language \textit{S}.
Applying this program to an input will produce an output by performing the computations described in the program.

% TODO: Improve typesetting of equations?
\begin{align*}
  \mathtt{output}\ &=\ \llbracket \mathtt{source} \rrbracket_{\mathtt{S}}\ \mathtt{input}
\end{align*}

Since a programming language \textit{S} usually cannot be executed directly by the underlying hardware, we can use an interpreter written in another language \textit{L} to perform the calculations.
This interpreter then accepts the source program as well as the original input to produce an output.

\begin{align*}
  \mathtt{output}\ &=\ \llbracket \mathtt{interpreter} \rrbracket_{\mathtt{L}}\ [\mathtt{source}, \mathtt{input}]
\end{align*}

Alternatively a compiler can be used, that first accepts the source program to produce a target program, which then can be applied to the original input.

\begin{align*}
  \mathtt{output}\ &=\ \llbracket \llbracket \mathtt{compiler} \rrbracket_{\mathtt{L}}\ \mathtt{source} \rrbracket_{\mathtt{T}}\ \mathtt{input} \\
  &=\ \llbracket \mathtt{target} \rrbracket_{\mathtt{T}}\ \mathtt{input}
\end{align*}

The nested brackets make it obvious, that multiple stages of computation are present. % TODO


%%% Local Variables:
%%% mode: latex
%%% TeX-master: "../paper"
%%% End:
