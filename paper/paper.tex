% ----------------------------------------------------------------------------
% Masterseminar
% ----------------------------------------------------------------------------
\documentclass[leqno,english]{scrarticle}

% =======================================================
\newcommand{\MyName}{Niklas Deworetzki}      % Name
\newcommand{\MyTitle}{MS5001~--~Masters Seminar}
\newcommand{\MyTopic}{Using Partial Evaluation to Generate Compilers}
% =======================================================

\usepackage[ngerman,main=english]{babel}
\usepackage[babel]{csquotes}         % Anführungszeichen

\usepackage[T1]{fontenc}             % Umlaute und Sonderzeichen
\usepackage[utf8]{inputenc}          % für Eingabe und Ausgabe
\usepackage{scrhack}                 % unterdrückt Fehlermeldung von listings

\usepackage{amsmath}                 % Mathematik
\usepackage{amssymb}                 % Mathematische Symbole
\usepackage{stmaryrd}
\usepackage{graphicx}                % Grafiken
\graphicspath{{images/}}             % Bilder werden aus images geladen
\usepackage{booktabs}
\usepackage{wrapfig}                 % Text neben Figuren
\usepackage{multicol}                % Mehrteilige Seiten

% Formatierungen für Grafiken und Elemente.
\usepackage[export]{adjustbox}
\usepackage[a4paper]{geometry}
\usepackage[titles,subfigure]{tocloft}
\usepackage{float}

% Formatierungen für Text, Überschriften, Inhaltsverzeichnis, etc.
\usepackage{fancyhdr}
\usepackage{tocbibind}
\usepackage{blindtext}

\usepackage[%
      rm={oldstyle=false,proportional=true},%
      sf={oldstyle=false,proportional=true},%
      tt={oldstyle=false,proportional=true,variable=false},%
      qt=false%
    ]{cfr-lm}

\usepackage[pdftex,
pdfauthor={\MyName{}},
pdftitle={\MyTitle{}~--~\MyTopic{}}]{hyperref}

% BibLatex configuration
\usepackage[backend=bibtex]{biblatex}
\usepackage[toc,acronyms]{glossaries}
\bibliography{bibliography.bib}     % Pfad zur Datei, die Referenzen enthält

% Verschiedene Grafiken, Figuren, Listings
\usepackage[table]{xcolor} % Definiere eigene Farben
\usepackage{textcomp}      % Sonderzeichen (für Operatoren in Code, z.B.)
\usepackage{subfigure}     % Teilfiguren
\usepackage{filecontents}  % Lade Dateien (für Codebeispiele)
\usepackage{listingsutf8}  % Codeblöcke mit UTF-8 (Umlaute)
\usepackage{lstautogobble} % Entfernt führende Leerzeichen in Codeblöcken
% lstautogobble=false zum deaktivieren für einzelne Codeblöcke

% Grafikbibliothek zum Erstellen eigener Diagramme
\usepackage{tikz}
\usetikzlibrary{decorations.pathreplacing,shapes,arrows,positioning}


%------------------------------------------------------------------------------
% THM Farbdefinitionen
\definecolor{cdmain1}{RGB}{128, 186, 36}
\definecolor{cdmain2}{RGB}{74, 92, 102}
\definecolor{cdhighlight1}{RGB}{156, 19, 46}
\definecolor{cdhighlight2}{RGB}{244, 170, 0}
\definecolor{cdhighlight3}{RGB}{0, 184, 228}
\definecolor{cdhighlight4}{RGB}{0, 40, 120}
%------------------------------------------------------------------------------
\colorlet{fontcolor}{cdmain2}
\colorlet{headercolor}{cdmain1}
%------------------------------------------------------------------------------
% Schrift für Fußnoten
\renewcommand{\footnotesize}{\fontsize{10pt}{12pt}\selectfont}
%------------------------------------------------------------------------------
% Format für Tabellen
\newcommand\HeaderCell[1]{%
  \multicolumn{1}{c}{\cellcolor{cdmain2}\textcolor{white}{#1}}
}
%------------------------------------------------------------------------------
% Farbdefinitionen für Codeblöcke
\colorlet{codecomment}{cdmain1}
\colorlet{codenormal}{cdmain2}
\colorlet{codelblue}{cdhighlight3}
\colorlet{codekeyword}{cdhighlight1}
\definecolor{backcolor}{rgb}{0.95,0.95,0.92}

\lstdefinestyle{codestyle}{
  backgroundcolor=\color{backcolor},
  commentstyle=\color{codecomment},
  keywordstyle=\color{codekeyword},
  numberstyle=\tiny\color{codenormal},
  stringstyle=\color{codelblue},
  basicstyle=\ttfamily\footnotesize,
  breakatwhitespace=false,
  breaklines=true,
  captionpos=b,
  keepspaces=true,
  numbers=left,
  numbersep=5pt,
  showspaces=false,
  showstringspaces=false,
  showtabs=false,
  tabsize=2,
  frame=single
}
\lstset{style=codestyle}

\include{glossary}
%------------------------------------------------------------------------------

%------------------------------------------------------------------------------
% Zeilenumbrüche in Links
\expandafter\def\expandafter\UrlBreaks\expandafter{\UrlBreaks\do\-\do\\/}
% Zeilenumbrüche bei Unterstrichen (in langen Variablennamen)
\renewcommand\_{\textunderscore\allowbreak}

\newcommand{\citationneeded}[1][]{\textsuperscript{\textcolor{red}{\textbf{[CN]}}}}
% ------------------------------------------------------------------------------

%==============================================================================
\begin{document}

\begin{titlepage}
  \begin{center}
    \begin{figure}
      % THM Logo
      \includegraphics[width=\textwidth]{LOGO_THM_CG_FB06}
    \end{figure}
  \end{center}
  % Title
  \Large
  \begin{center}
    \vspace{1cm}
    \textbf{\MyTitle{}}\linebreak
    \vspace{1cm}
  \end{center}
  \large
  Subject:\\
  \textbf{\MyTopic{}}

  \normalsize
  \vfill
  % \begin{tabular*}{\textwidth}[t]{l,c,l}
  \begin{center}
    \begin{tabular*}{0.75\textwidth}%
      {@{\extracolsep{\fill}}ll}

      {Submitted by:} & {\MyName{}}\\
      {} & {\href{mailto:niklas.deworetzki@mni.thm.de}{niklas.deworetzki@mni.thm.de}}\\
      {} & {}\\
      {Matriculation number:} & {1234567}\\ % TODO
      {} & {}\\
      {} & {}\\
      {Tutor:} & {Prof.\ Dr.\ Uwe\ Meyer}\\
      {} & {}\\
      {} & {}\\
      {Date of Submission:} & {\today}

    \end{tabular*}
  \end{center}

  \vfill
\end{titlepage}



\section{Introduction}\label{sec:introduction}

PE captures many forms of automated generation and analysis of programs.
optimization, interpretation and automatic generation, as compilers or more general program generators.

Base use case is to specialize programs.
Specialization similar to currying (Logic, FP) or projection (mathematics).
Idea behind this concepts is to reduce a many argument function to single argument function, by fixing parameters.
Specialization yields same result for programs, that accept many inputs.
Some parts of input are known statically, specialization generates program requiring remaining inputs.

THeoretical base is s-n-m theorem\citationneeded[Kleene].
Important work computability, no attention to efficiency/speed.
In computation/praxis, intuitively specialized programs can be faster.
Aim for practical specialization.

\subsection{Terminology}

Program performing PE is called PE or specializer.
Sometimes called Mix, since mixes computation (see chapter 2).
Also mixes languages, implementation language, source language, target language, as seen later.
Not explicitly mentioned, language is not relevant. Useful to keep in mind, that boundaries exist.
Sometimes not clear with terminology, since PE specializes programs and not just simple functions.
Since single functions can be turned into programs, terminology might change with context of explanation.
Another point: Programs and computability are intertwined, we assume programs always halt for simplicity, unless noted.
Must be considered, when implementing or real world application.

\subsection{Structure of this paper}




%%% Local Variables:
%%% mode: latex
%%% TeX-master: "../paper"
%%% End:


\section{Execution of Programs}\label{chap:programs}

Usually, we think of the execution of programs as one single step.
Either a program is executed, producing an output to its given inputs, or a program sits on a hard drive without being executed.
This way, we don't have to worry about the exact details of how a program is executed and can focus on the computational problem a program tries to solve.
However, in the following chapters, we will look into many ways to execute a program, all of which lead to the same output but are drastically different in their use cases, advantages and behaviors.

\begin{figure}[h]
  \centering
  \begin{tikzpicture}[
    base/.style={draw, thick, minimum width = 2.4cm, minimum height = 1cm},
    data/.style={rectangle, base},
    program/.style={ellipse, base},
    arrow/.style={-triangle 90}]

    \node[program] (prog) at (0, 0) {program};
    \node[data, right = of prog] (output) {output};
    \node[data, left = of prog] (input2) {input\textsubscript{2}};
    \node[data, above = 0cm of input2] (input1) {input\textsubscript{1}};
    \node[data, below = 0cm of input2] (input3) {$\ldots$};

    \draw [arrow] (input1.east) -- (prog);
    \draw [arrow] (input2.east) -- (prog);
    \draw [arrow] (input3.east) -- (prog);
    \draw [arrow] (prog) -- (output);
  \end{tikzpicture}
  \caption{Direct execution of a program}\label{fig:native-program}
\end{figure}


Figure~\ref{fig:native-program} schematically presents the way, we usually think of the execution of a program.
A program (described by an oval shape) accepts some data (described by boxes) as an input and transforms this data into an output.
While this diagram conceals details of how execution actually takes place, it gives an appropriate intuition about the general flow of data during execution.
The conventional execution of a program takes place as a single computational step.
Even if a program gets more complex, accepting many different inputs and producing output for each of these cases, it is only the run-time that grows, not the computational stages.


The figure, even if useful as an introduction, does not correctly represent reality in most cases.
Usually, a program written by a programmer is not executable directly on a computer and the help of other programs is required to transform the raw source code into an executable format.
These programs are so-called meta-programs, as the datasets they work on are programs as well.
In the context of this work, we will come into contact with three different meta-programs that are used for the execution of other programs, namely interpreters, compilers and partial evaluators.
While interpreters and compilers are usually familiar, the concept of a partial evaluator often seems foreign.
To put this concept in a better perspective, we will start with a brief description of interpreters and compilers before partial evaluators are finally introduced.


\subsection{Interpreters}

Interpreters, as already mentioned are meta-programs that are used to execute other programs.
An interpreter accepts the program, which is to be executed, as an input and analyzes it, whereby it is usually transformed into an intermediate representation.
This tree-like structure mirrors the structure and contents of the source program.
By assigning meaning to every node of this tree, a program becomes executable, as the tree can be traversed with the interpreter performing actions corresponding to the meaning of each node.

Now, given that the interpreter is executable, all programs written in a language the interpreter can \enquote{understand} are executable too.
As Figure~\ref{fig:interpreted-program} shows, the interpreted execution of a program looks very similar to the direct, native execution.
The only difference is, that the program to be executed is now itself passed as an input to the interpreter along with its inputs.
Execution this way is still a single computational stage.

Interpreters provide a simple solution to execute programs in another language and are relatively easy to implement.
On the downside, execution via an interpreter provides less performance, as interpreter and interpreted program share the same resources and some computational overhead is required for analysis of said program.

\begin{figure}[h]
  \centering
  \begin{tikzpicture}[
    base/.style={draw, thick, minimum width = 2.4cm, minimum height = 1cm},
    data/.style={rectangle, base},
    program/.style={ellipse, base},
    arrow/.style={-triangle 90}]

    \node[program] (interpreter) at (0, 0) {interpreter};

    \node[data, above left = -0.4cm and 1.4cm of interpreter] (input) {input\textsubscript{1}};
    \node[data, below = 0cm of input] (input2) {input\textsubscript{2}};
    \node[data, above = 0cm of input] (program) {program}; \node[program, above = 0cm of input] {};
    \node[data, below = 0cm of input2] (input3) {$\ldots$};

    \node[data, right = of interpreter] (output) {output};

    \draw [arrow] (program.east) -- (interpreter);
    \draw [arrow] (input.east) -- (interpreter);
    \draw [arrow] (input2.east) -- (interpreter);
    \draw [arrow] (input3.east) -- (interpreter);
    \draw [arrow] (interpreter) -- (output);
  \end{tikzpicture}
  \caption{An interpreter executing a program}\label{fig:interpreted-program}
\end{figure}


\subsection{Compilers}

Compilers, similar to interpreters, are programs that are used to transform other programs.
While an interpreter uses the intermediate representation of a program to directly executed the assigned meaning to it, a compiler translates the meaning into another language.
As a result, a compiler produces a single output, which is the so-called target program.
This program holds the meaning of the compiler's input~--~the source program~--~translated into another language.
As seen in Figure~\ref{fig:compiled-program}, an additional stage of computation is introduced this way.
The target program has to be executed for it to accept inputs and produce an output like the original source program.

As already apparent visually, the translation process of a compiler is more complex than the process of an interpreter.
But since the input program has to be only analyzed once and the target program is executed as a standalone program, compilation usually provides better performance in contrast to interpretation.
This is especially true if the same program is executed multiple times since no computational overhead is present during the execution of the target program after it has been created once.

\begin{figure}
  \centering
  \begin{tikzpicture}[
    base/.style={draw, thick, minimum width = 2.4cm, minimum height = 1cm},
    data/.style={rectangle, base},
    program/.style={ellipse, base},
    arrow/.style={-triangle 90}]

    \node[program] (compiler) at (0, 0) {compiler};
    \node[data, left = of compiler] (source) {source};
    \node[data, below = of compiler] (target) {target}; \node[program, below = of compiler] {};

    \node[data, left = of target] (input) {input};
    \node[data, right = of target] (output) {output};

    \draw [arrow] (source) -- (compiler);
    \draw [arrow] (compiler) -- (target);

    \draw [arrow] (input) -- (target);
    \draw [arrow] (target) -- (output);
  \end{tikzpicture}
  \caption{A compiler generating an executable target program}\label{fig:compiled-program}
\end{figure}



\subsection{Partial Evaluators}

Partial evaluators, as the third meta-program introduced and focus of this paper, generalize the previously mentioned concept of creating additional computational stages.
The inputs of a partial evaluator are a source program as well as \textit{some} inputs for this program.
Given these inputs, a partial evaluator will perform all computations in the source program that are available under the given inputs.
As not all inputs for the source program are present, some computations cannot be performed.
These remaining computations form a so-called residual program, the output of a partial evaluator.

\begin{figure}[h]
  \centering
  \begin{tikzpicture}[
    base/.style={draw, thick, minimum width = 2.4cm, minimum height = 1cm},
    data/.style={rectangle, base},
    program/.style={ellipse, base},
    arrow/.style={-triangle 90}]

    \node[program] (pe) at (0, 0) {pe};
    \node[data, left = of pe] (input1) {input\textsubscript{1}};
    \node[data, above = 0cm of input1] (source) {source};
    \node[data, below = of pe] (residual) {residual}; \node[program, below = of pe] {};

    \node[data, left = of residual] (input2) {input2};
    \node[data, below = 0cm of input2] (input3) {$\ldots$};
    \node[data, right = of residual] (output) {output};

    \draw [arrow] (source.east) -- (pe);
    \draw [arrow] (pe) -- (residual);

    \draw [arrow] (input1.east) -- (pe);
    \draw [arrow] (input2.east) -- (residual);
    \draw [arrow] (input3.east) -- (residual);

    \draw [arrow] (residual) -- (output);
  \end{tikzpicture}
  \caption{A partial evaluator generating a residual program}\label{fig:partial-evaluated-program}
\end{figure}

While Figure~\ref{fig:partial-evaluated-program}, describing a partial evaluator, is structurally similar to Figure~\ref{fig:compiled-program} describing a compiler, the concept is generalized.
A compiler only splits the overhead of analysis and the actual computation into different computational stages.
A partial evaluator can split calculations depending on different inputs into different computational stages.
This way, a general program accepting multiple inputs can be specialized to a fixed input, which usually leads to higher performance, since the overhead of recognizing inputs is removed.



%%% Local Variables:
%%% mode: latex
%%% TeX-master: "../paper"
%%% End:


\section{Partial Evaluation}\label{sec:partial-evaluation}

Overview of inner workings of partial evaluators.
Already covered: PE accepts a source program as input and additionally inputs for the source program itself.
Source program is analyzed and structured, to allow for later transformation
Section explains different variants of PEs and how they decide their ``targets'' (computable stuff)
Afterwards different methods/tools/intruments of evaluations are explained.


\subsection{Offline and Online Partial Evaluation}\label{sec:offline-vs-online}

Two main variants on partial evaluators.
Both act on the structured source program as input data.
Both require some known input data for the source.
Difference is in how computations are decided.

To decide what to compute: Static vs. Dynamic.
Static means (similar to compiler), that it depends on fixed values\citationneeded[Does it?]
Since PE knows input for program, it can find static computations
All static computations can be performed by PE.
Dynamic computations are generated for runtime.

The decision of static vs dynamic is called division.
Every part of program has to recognized either static or dynamic.
How this division is made is different in the two main variants \citationneeded[A Hybrid Approach to Online and Offline Partial Evaluation].


Offline makes decision before specialization in a preprocessing phase called bindingtime analysis(BTA).
Conservative as everything is treated dynamic until proven static.
Annotates program \citationneeded[Jones, Chap 7] without knowing concrete values of static input.
Then transforms input.

Online makes decision during specialization with the value of a static input.
Is more complex, since binding time analysis is not factored out as separate preprocessing step.
Harder to predict speedup, difficult to self application or guarantee termination.
Traverses source code deciding on the fly, if static or dynamic.
Can use the actual values of input to perform decision, but may encounter same code multiple times.


\subsection{Instruments of Partial Evaluation}\label{sec:pe-instruments}

Overview what techniques are available for PE to compute static.
Mainly PE is based on the propagation of static values.
Since it allows to unlock more computations.

Constant propagation (sparse constant propagation).
constants (input) are known and computed.
Expressions depending on constants can be computed too.
Also conditions -> control flow is decided statically.
Allows to cut off unused paths in code, perform many calculations.

\begin{lstlisting}
  x := 2, n := 8
  def power(int n, int x) {
    int result = 1;
    while (n > 1) {
      result = result * x;
    }
  }
\end{lstlisting}


Unfolding (of function calls)
Allows to propagate constants into called procedures.
Or unroll loops into a sequence of computations.

\begin{lstlisting}
  def power(int n, int x) {
    if (n == 0) {
      return 1;
    } else {
      return x * power(n - 1, x);
    }
  }
\end{lstlisting}


symbolic computation
Uses statically known structure of expressions to deduce simplifications.
E.g mathematic simplicitations based on structure or mathematic identities.

\begin{lstlisting}
  2 << n
\end{lstlisting}



%%% Local Variables:
%%% mode: latex
%%% TeX-master: "../paper"
%%% End:


\section{The Futamura-Projections}\label{sec:futamura}

Until now, partial evaluation was only presented as a tool to specialize and at the same time optimize programs.
The specialized programs shown did not have any new or interesting properties, except that they were faster than their general counterparts.

In \citationneeded[] Futamura showed, that it is not only possible to use partial evaluation for specialization of programs or to divide a computation into multiple stages.
He proposed in total three projections, now known as the Three Futamura Projections, in which partial evaluators create programs with interesting and seemingly new properties.
In the meantime, these projections have been confirmed multiple times by existing partial evaluators.
This does not seem surprising, considering that only a partial evaluator and an interpreter is required for the three projections.

Next, before elaborating on the projections, a new notation is introduced.
This notation is used to describe the behavior of programs and computations in multiple stages.


\subsection{Notation of Multistage Computations}\label{sec:multistage-notation}

As seen before, computations can be split into multiple stages, performed by multiple programs, that accept multiple inputs (which themself can be other programs).
While the notation from Chapter~\ref{chap:programs} might be useful as an visual introduction, it lacks the brevity for more advanced explanations.
The following equation serves as an example to introduce the new notation:

\begin{align*}
  \mathtt{output}\ =\
  \llbracket \mathtt{p} \rrbracket\ (\mathtt{input_1},\, \mathtt{input_2},\, \mathtt{input_3})
\end{align*}

It shows a simple program called \texttt{p}, that accepts three inputs named \texttt{input\textsubscript{1}} to \texttt{input\textsubscript{3}}.
The output of this program is aptly named \texttt{output} and (as well as the three inputs) may be data or an executable program.
The actual execution of \texttt{p} is shown by the square brackets, that simultaneously separate different computational stages.
This property becomes clear, when the operation of a compiler is described as an equation this way:

\begin{align*}
  \mathtt{output}\
  &=\ \llbracket \mathtt{target} \rrbracket \mathtt{_T}\ (\mathtt{input}) \\
  &=\ \llbracket \mathtt{source} \rrbracket \mathtt{_S}\ (\mathtt{input}) \\
  &=\ \left\llbracket \llbracket \mathtt{compiler} \rrbracket \mathtt{_I}\ (\mathtt{source}) \right\rrbracket\mathtt{_T}\ (\mathtt{input})
\end{align*}

The output produced by executing a target program is the same as if a source program were to be executed directly.
Also the same output is produced, by compiling the source program and executing the compilers output.
Two things are important to note here: % TODO: Is it required to show the languages?
Firstly, the brackets to describe evaluation can be annotated by a subscript, describing the language that is used to determine the meaning of a program.
The compiler is executed according to the meaning of its implementation language \texttt{I}, while source is a program written in a source language \texttt{S}, that is translated into a target language \texttt{T} by the compiler.
Additionally it is important to note, what equality means in this context.
While the output produced by the right-hand sides of this equation is the same, the details of how this output is produced in each case can vary widely.


\subsection{The First Futamura Projection}\label{sec:futamura-first}

The first of Futamura's projections is based on a fact, that was already shown in this paper but was not directly emphasized.
As we already know, an interpreter accepts two inputs: A program to execute and the input for this program itself.
Usually these inputs are passed in at once, resulting in one computational stage.
But a partial evaluator (\texttt{pe}) could be used to separate these inputs, as shown in the following equation.

\begin{align}
  \mathtt{output}\
  &=\ \llbracket \mathtt{source} \rrbracket \ (\mathtt{input}) \\
  &=\ \llbracket \mathtt{interpreter} \rrbracket \ (\mathtt{source},\, \mathtt{input}) \\
  &=\ \llbracket \llbracket \mathtt{pe} \rrbracket \ (\mathtt{interpreter},\, \mathtt{source}) \rrbracket \ (\mathtt{input}) \label{eqn:pe-fut-1} \\
  &=\ \llbracket \mathtt{target} \rrbracket \ (\mathtt{input})
\end{align}

It might not become obvious at first.
But looking at equation~(\ref{eqn:pe-fut-1}) one may realize, that $\llbracket \mathtt{pe} \rrbracket \ (\mathtt{interpreter},\, \mathtt{source})$ creates a program with the same meaning as \texttt{source} itself.
The difference, however, is that while \texttt{source} is a program written in the source language, the residual program is written in the output language of the partial evaluator.

Notably the partial evaluator acted like a compiler and the act of specialization produced a compiled program.


\subsection{The Second Futamura Projection}\label{sec:futamura-second}

The second of Futamura's projections is based on the first.
Looking at the second projection, it becomes clear that the same scheme can be applied again.
The partial evaluator itself is a program that accepts two inputs: an interpreter and a source program.

So for the second projection, a partial evaluator is used, to create another stage of computation, abstract away the source as an input parameter.
The following equation describes a partial evaluator that is used to specialize a partial evaluator in respect to an interpreter.

\begin{align}
  \mathtt{target}\
  &=\ \llbracket \mathtt{pe} \rrbracket \ (\mathtt{interpreter},\, \mathtt{source}) \\
  &=\ \llbracket \llbracket \mathtt{pe} \rrbracket \ (\mathtt{pe},\, \mathtt{interpreter}) \rrbracket \ (\mathtt{source}) \label{eqn:pe-fut-2}\\
  &=\ \llbracket \mathtt{compiler} \rrbracket \ (\mathtt{source})
\end{align}

In this case, the residual program created in equation~(\ref{eqn:pe-fut-2}) is a program, that can transform a source program into a target program.
Consequently, using the partial evaluator and an interpreter, it is possible to create a compiler, that can translate arbitrary other programs.
Furthermore the partial evaluator acted like a compiler generator, creating a compiler from nothing more than a partial evaluator and an interpreter.


\subsection{The Third Futamura Projection}\label{sec:futamura-thrid}

The third of Futamura's projections uses the same scheme as the previous two.
This time, it is the partial evaluator, specializing the partial evaluator in respect to an interpreter, that is the program accepting two inputs.
Again it is possible to separate these two inputs by introducing another computational stage.
The following equation describes a partial evaluator that is used to specialize a partial evaluator in respect to an partial evaluator.

\begin{align}
  \mathtt{compiler}\
  &=\ \llbracket \mathtt{pe} \rrbracket \ (\mathtt{pe},\, \mathtt{interpreter}) \\
  &=\ \llbracket \llbracket \mathtt{pe} \rrbracket \ (\mathtt{pe},\, \mathtt{pe}) \rrbracket \ (\mathtt{interpreter}) \label{eqn:pe-fut-3}\\
  &=\ \llbracket \mathtt{compiler\operatorname{-}gen} \rrbracket \ (\mathtt{interpreter})
\end{align}

The residual program created in the third Futamura Projection in equation~(\ref{eqn:pe-fut-3}) is a program that can generate a compiler.
This program is a compiler generator, that accepts the description of a languages semantic to generate a standalone compiler.
The description is passed as an interpreter, which decides the behavior of the generated compiler.


\subsection{Is there a Fourth Futamura Projection?}\label{sec:futamura-fourth}

The equations emerging from the third Futamura Projection still have the same structure as the previous equations, which would allow further to apply the previous abstraction scheme.
It is notable, however, that further applications do not change the resulting equations.
While this property may be called the Fourth Futamura projection, the equations themselves do not bear any new insights.

\begin{align}
  \mathtt{compiler\operatorname{-}gen}\
  &=\ \llbracket \mathtt{pe} \rrbracket \ (\mathtt{pe},\, \mathtt{pe}) \\
  &=\ \llbracket \llbracket \mathtt{pe} \rrbracket \ (\mathtt{pe},\, \mathtt{pe}) \rrbracket \ (\mathtt{pe}) \\
  &=\ \ldots \nonumber
\end{align}



\subsection{Going Further}\label{sec:self-application}

While no new functionality or properties arise from further application of the above scheme, it may still be desirable to further apply partial evaluation.
The key insight is, that there are multiple ways to create a program with some specific behavior.
But while these programs behave equally on a theoretical level as it is shown in equations, there are differences in practice.
As it was shown with existing implementations, it is possible to gain significant speed-ups of a program, if a \enquote{specialized} compiler or compiler-generator was used.

Another interesting property, that becomes clear through Futamura's projections is the relationship between programs and their generating extension.
A generating extension of a program $p$ is a program $\mathtt{pe}_p$ that accepts an input $i$ to produce a version of $p$ that is specialized in respect to $i$ \citationneeded[Fourth Projection, 4.2].
A good example for a generating extension can be seen in the second Futamura Projection:
The compiler $\llbracket \mathtt{pe} \rrbracket \ (\mathtt{pe},\, \mathtt{interpreter})$ is a generating extension of the interpreter, since it accepts a source program and specializes the interpreter in respect to this source program.
It turns out that compiler generators represent the generating extension of partial evaluators, which indicates a strong connection between them.


\begin{table}[h]
  \centering
  \begin{tabular}{l l} % TODO: Spacing here
    \toprule
    program & generating extension \\
    \midrule
    $\llbracket \mathtt{interpreter} \rrbracket \ (\mathtt{source},\, \mathtt{input})$
            & $\llbracket \llbracket \mathtt{compiler} \rrbracket \ (\mathtt{source}) \rrbracket \ (\mathtt{input}) $\\
    $\llbracket \mathtt{parser} \rrbracket \ (\mathtt{grammer},\, \mathtt{text})$
            & $\llbracket \llbracket \mathtt{parser\operatorname{-}gen} \rrbracket \ (\mathtt{grammar}) \rrbracket \ (\mathtt{text})$ \\
    $\llbracket \mathtt{pe} \rrbracket \ (\mathtt{interpreter}, \mathtt{source})$
            & $\llbracket \llbracket \mathtt{compiler\operatorname{-}gen} \rrbracket \ (\mathtt{interpreter}) \rrbracket \ (\mathtt{source})$ \\
    \bottomrule
  \end{tabular}
  \caption{Programs and their generating extensions}\label{tab:generating-extensions}
\end{table}

%%% Local Variables:
%%% mode: latex
%%% TeX-master: "../paper"
%%% End:


\section{Critical Assessment}\label{sec:discussion}

PE is great technology to solve many problems.

Advantageous in Software Development.
General programs can be fast, speed-up.
Why not widespread, could tooling/compilcated use.


Not ripe in some areas.
Online is fast, but has termination problems.
Offline has less problems, but less speed up.
Generally speed-up is hard to predict/dependent on input.


Use as compiler is nice in theory, not really in practice.
Wont replace classical compiler construction.
Compilers are not created ``from nothing'', interpreters must exist and PEs too.
Target language is language from PE, for some tricks limited in language.
Compiler constructors would be needed in those fields, developing PEs.

Even existing PE and interpreter wont completely replace CC.
For optimizing compilers, as industry requires, clever people are needed.
PEs can't invent new data structures, can't invent mathematics, simplifications or dirty tricks.
PE only restructure input using known rules.
People are needed to implement or invent those rules,
people are needed to keep up to date, since technology is evolving.



%%% Local Variables:
%%% mode: latex
%%% TeX-master: "../paper"
%%% End:


\end{document}

%%% Local Variables:
%%% mode: latex
%%% TeX-master: t
%%% End:
